% dhgefigure -> ...
\DeclareDocumentCommand{\dhgefigure}{O{h} m m m O{} O{}}
{
	\begin{figure}[#1]
		\begin{center}
			\includegraphics[#3]{#2}
		\end{center}
		\caption{#4}
	
		\ifx #5\empty \else
			\ifx #6\empty \else
				{\small \protect \textbf{Quelle:} \cite[#6]{#5}}
			\fi
		\fi
	
	\end{figure}
}

\newcommand{\markBox}[2]
{
    \def\default{#2 {$\square$} #2 {$\square$} #2 {$\square$} #2 {$\square$}}

    \ifnum#1 = 1
    \def\default{#2 {$\boxtimes$} #2 {$\square$} #2 {$\square$} #2 {$\square$}}
    \else
        \ifnum#1 = 2
        \def\default{#2 {$\square$} #2 {$\boxtimes$} #2 {$\square$} #2 {$\square$}}
        \else
            \ifnum#1 = 3
            \def\default{#2 {$\square$} #2 {$\square$} #2 {$\boxtimes$} #2 {$\square$}}
            \else
                \ifnum#1 = 4
                \def\default{#2 {$\square$} #2 {$\square$} #2 {$\square$} #2 {$\boxtimes$}}
                \fi
            \fi
        \fi
    \fi
    \hspace*{-.5cm}\default
}

% new command -> \doubleunderline
\newcommand{\doubleunderline}[1]{
	\underline{\underline{#1}}
}
