\usepackage		[a4paper,
    	            inner  = 3cm,
                    outer  = 2cm,
                    top    = 2.5cm,
                    bottom = 2.5cm]{geometry}
\usepackage		{setspace}
\usepackage		[hyperfootnotes = false,
                    hidelinks]{hyperref}
\usepackage		{amssymb}
\usepackage		{fancyhdr}
\usepackage		[version = 3]{acro}
\usepackage		{enumitem}
\usepackage		[style=german]{csquotes}
\usepackage		[backend=biber,
                    style        = alphabetic,
                    citestyle    = components/alphabetic-ibid,
                    giveninits   = true,
                    ibidtracker  = true,
                    minbibnames  = 3,
                    minalphanames= 3]{biblatex}
\usepackage		[ngerman]{babel}
\usepackage		{footmisc}
\usepackage		{graphicx}
\usepackage		{caption}
\usepackage		{xparse}
\usepackage		{float}
\usepackage		{tocloft}
\usepackage     {lmodern}
\usepackage     {totcount}
\usepackage     {chngcntr}

% schönere Fonts, aber optional. Zum deaktivieren CFANCYFONTS in config.tex auf 0 setzen
\if\CFANCYFONTS 1
\usepackage     [scaled=0.88]{beraserif} % Bera Serifen Font
\usepackage     [scaled=0.85]{berasans} % Bera Sans Font
\usepackage     [scaled=0.84]{beramono} % Bera Mono Font
\usepackage     [T1]{fontenc}
\usepackage     {mathpazo} % Palatino Font
\usepackage     [T1,small,euler-digits]{eulervm} % Euler Font
\usepackage     {listings}
\lstset{
	basicstyle=\ttfamily,
    breaklines=true
}
\fi

% VERALTETE PAKETE
% nicht mehr benötigt, aber für Rückwärtskompatibilität noch enthalten
% einkommentieren, wenn nach einem Template-Update Probleme auftauchen

%\usepackage		[utf8]{inputenc}
%\usepackage		{titletoc}
%\usepackage		{csquotes,xpatch}
%\usepackage		{ifthen}
%\usepackage		{etoolbox}
%\usepackage     {icomma}
