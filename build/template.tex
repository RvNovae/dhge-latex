% ROOT & PACKAGE SETUP
\documentclass[a4paper, 12pt]{article}

% Präambel laden

\input{../preamble.tex}

% Nutzerkonfiguration laden
% Projektarbeit Nr. (1 bis 4) oder Bachelorarbeit (B)
\def\CARBEIT        {1}

% Title der Arbeit
\def\CTITLE         {THEMA}

% Author der Arbeit (ANR -> Anrede, VOR -> Vorname, NACH -> Nachname)
\def\CAUTHORANR     {0}     % 1 -> Frau !1 -> Herr
\def\CAUTHORVOR     {VOR}
\def\CAUTHORNACH    {NACH}

% "vorlege am" - Datum
\def\CDATUM         {\today}

% Matrikelnummer des Authors
\def\CMATRIKEL      {MTR-NR}

% Kurs des Authors
\def\CKURS          {KURS}

% DHGE Campus des Authors (Gera/Eisenach)
\def\CCAMPUS        {Gera}

% Studienbereich des Authors
\def\CBEREICH       {Technik}

% Studiengang des Authors
\def\CSTUDIENGANG   {STUDIENGANG}

% Betrieb des Authors (nur der Name des Betriebs keine Adresse)
\def\CBETRIEB       {FIRMA}

% Betreuer der Arbeit (den akademischen Titel nicht vergessen)
\def\CBETREUER      {BETREUER}

% Fügt einen einfachen Sperrvermerk hinter das Deckblatt (1 = aktiv)
\def\CSPERRVERMERK  {0}

% verwende Richtlinien nach Prof. Dr. Kusche
% zum Aktivieren auf 1 setzen
\def\CKUSCHE        {0}

% 1 aktiviert das Einbinden des Abstracts
\def\CHASABSTRACT   {0}

% 1 aktiviert den Font-Vorschlag, 0 deaktiviert ihn
\def\CFANCYFONTS    {1}

% 1 setzt Absatztrenner auf Einrückungen, 0 auf vertikale Abstände
\def\CEINR          {0}


% verwendete Pakete laden und konfigurieren
\usepackage		[a4paper, 
    	         inner  = 3cm, 
                 outer  = 2.5cm, 
                 top    = 2.5cm, 
                 bottom = 2.5cm]{geometry}
\usepackage		[utf8]{inputenc}
\usepackage		{setspace}
\usepackage		{titletoc}
\usepackage		[hyperfootnotes = false, 
                 hidelinks]{hyperref}
\usepackage		{amssymb}
\usepackage		{fancyhdr}
\usepackage		[version = 3]{acro}
\usepackage		{enumitem}
\usepackage		[T1]{fontenc}
\usepackage		[style=german]{csquotes}
\usepackage		[backend=biber, 
                 style        = alphabetic,
                 citestyle    = components/alphabetic-ibid, 
                 giveninits   = true, 
                 ibidtracker  = true,
                 minbibnames  = 3,
                 minalphanames= 3]{biblatex}
\usepackage		[ngerman]{babel}
\usepackage		{csquotes,xpatch}
\usepackage		{footmisc}
\usepackage		{graphicx}
\usepackage		{caption}
\usepackage		{ifthen}
\usepackage		{xparse}
\usepackage		{float}
\usepackage		{etoolbox}
\usepackage		{tocloft}
\usepackage     {icomma}
\usepackage     {lmodern}
\usepackage     {totcount}


% Setup von Commands und Dokument
% dhgefigure -> ...
\DeclareDocumentCommand{\dhgefigure}{O{tbp} m m m m O{} O{}}
{
    \begin{figure}[#1]
        \begin{center}
            \includegraphics[#3]{#2}
        \end{center}
        \caption{#4}
        \label{#5}

        \ifx #6\empty \else
            \ifx #7\empty \else
                {\small \protect \textbf{Quelle:} \cite[#7]{#6}}
            \fi
        \fi

    \end{figure}
}

% SubSubSubSection
\newcommand{\dhgeparagraph}[1]{\paragraph{#1}\mbox{}\\\vspace{-1.5em}}

% Doppelte Unterstreichung
\newcommand{\doubleunderline}[1]{
    \underline{\underline{#1}}
}

% Formatierung der Bachelorarbeit: Autorreferat und Thesenblatt
\newcommand{\baFormat}[2]{
    \begin{center}
        {\LARGE\bf #1}

        \vspace{0.7cm}
        {\large\bf\enquote{\CTITLE}}

        \vspace{0.5cm}
        von \CAUTHOR
    \end{center}

    \vspace{1.5cm}

    {#2}

    \cleardoublepage
}

% => latex default root dir is the dir of the start file
%   => build/template.tex => root dir = build/
% FONT SETUP
% schönere Fonts, aber optional. Zum deaktivieren CFANCYFONTS in config.tex auf 0 setzen
\if\CFANCYFONTS 1
    \DeclareMathAlphabet{\mathrm}{OT1}{cmr}{m}{n} % mathrm soll weiter Computer Modern als Font nutzen, siehe #103
    \DeclareMathAlphabet{\mathit}{OT1}{cmr}{m}{it}
    \lstset{
        basicstyle=\ttfamily,
        breaklines=true
    }
\fi


% VARIABLE SETUP
% die veraltete CAUTHOR Variable wird automatisch befüllt
\def\CAUTHOR{\CAUTHORVOR\ \CAUTHORNACH}

\if\CARBEIT B
    \def\BETREUER{Gutachter}
\else
    \def\BETREUER{Betreuer}
\fi


% DOCUMENT SETUP
\onehalfspacing % 1.5 line spacing
% TODO: sollte theoretisch keine Verwendung mehr haben
% \widowpenalty10000
% \clubpenalty10000


% INHALTSVERZEICHNIS SETUP
\renewcommand{\contentsname}{Inhaltsverzeichnis}
\cftsetindents{section}{0em}{4em}
\cftsetindents{subsection}{0em}{4em}
\cftsetindents{subsubsection}{0em}{4em}
\setcounter{tocdepth}{3}
\setcounter{secnumdepth}{5}


% ABBILDUNGEN UND TABELLEN SETUP
\renewcommand{\listfigurename}{Abbildungsverzeichnis}
\renewcommand{\listtablename}{Tabellenverzeichnis}

\addto{\captionsngerman}{
    \renewcommand*{\figurename}{Abb.}
    \renewcommand*{\tablename}{Tab.}
}

\addtocontents{lof}{\linespread{2}\selectfont}
\addtocontents{lot}{\linespread{2}\selectfont}

\makeatletter
\renewcommand{\cftfigpresnum}{Abb. }
\renewcommand{\cfttabpresnum}{Tab. }

\setlength{\cftfignumwidth}{2cm}
\setlength{\cfttabnumwidth}{2cm}

\setlength{\cftfigindent}{0cm}
\setlength{\cfttabindent}{0cm}
\makeatother

% MODUS KUSCHE: ABBILDUNGEN UND TABELLEN SETUP
% im Kusche-Mode sollen Abbildungen nach Kapitel.lfd nummeriert werden
\if\CKUSCHE 1
    \counterwithin{figure}{section}
    \counterwithin{table}{section}
\fi


% CAPTION SETUP
\captionsetup{
    font = small,
    labelfont = bf,
    singlelinecheck = false,
    skip = 10pt,
    belowskip = 0pt
}


% CITATION SETUP
\renewcommand*{\labelalphaothers}{\textsuperscript{}}


% HEADERS & FOOTERS
\pagestyle		{fancyplain}
\fancyhf		{}
\renewcommand	{\headrulewidth}{0pt}
\renewcommand	{\footrulewidth}{0pt}
\setlength		{\headheight}{15pt}

% KUSCHE MODE: HEADERS & FOOTERS

\fancyfoot      [R]{\thepage} % nach den neuen Anforderungen sind Seitenzahlen unten rechts, geht d'accord mit dem Kusche-Mode
\if\CKUSCHE 1
    \fancyfoot      [L]{\leftmark} % im Kusche-Mode erscheint linksbündig das Kapitel in der Fußzeile
\fi


% FOOTNOTE SETUP
\renewcommand{\footnotelayout}{\hspace{0.5em}}

% COUNTER
% Zweck: in römischen Zahlen weiter zählen, nachdem der Counter von arabisch zurück geändert wird
\newcounter{savepage}


% SECTION SETUP
% sections sollen mit Seitenumbruch beginnen

\let\stdsection\section
\renewcommand\section{\newpage\stdsection}


% MATHRM ADJUSTMENTS
% Überschreiben von \mathrm{} -> einheitlichen Abstand einfügen
\let\oldMathrm\mathrm
\renewcommand{\mathrm}[1]{\,\oldMathrm{#1}}

% PATH SETUP
% root ist ist das neue Arbeitsverzeichnis
\makeatletter
\def\input@path{{../}{path1/}}
\makeatother

\graphicspath	{{../assets/img/}}


% AUTO REMOVE/INSERT ABBILDUNGSVERZEICHNIS & TABELLENVERZEICHNIS
% Conditionals um AbbildungVZ und TabellenVZ nur zu rendern, wenn sie nicht leer sind
\newtotcounter{figCount}
\newtotcounter{tabCount}
\let\oldTabTOC=\table
\let\oldFigTOC=\figure
\renewcommand{\figure}{\stepcounter{figCount}\oldFigTOC}
\renewcommand{\table}{\stepcounter{tabCount}\oldTabTOC}

\newcommand{\conditionalLoF}{
    \ifnum\totvalue{figCount}>0
        \addcontentsline{toc}{section}{\listfigurename}
        \listoffigures
        \cleardoublepage
    \fi
}
\newcommand{\conditionalLoT}{
    \ifnum\totvalue{tabCount}>0
        \addcontentsline{toc}{section}{\listtablename}
        \listoftables
        \cleardoublepage
    \fi
}


% ANLAGENVERZEICHNIS SETUP
% definiert eine neue Liste für das Anlagenverzeichnis
\newcommand{\listanlageverzeichnis}{\vspace*{-20pt}}
\newlistof{anlagen}{alt}{\listanlageverzeichnis}

% Befehl welcher ein Item dem Anlagenverzeichnis hinzufügt
\newcommand{\addItemToAnlageverzeichnis}[1]{%
    \def\fig{fig}
    \def\tab{tab}

    \ifx\fig\typeOfCap
        \def\type{\thefigure}
        \def\name{Abb.\hspace{8pt}}
    \else \ifx\tab\typeOfCap
            \def\type{\thetable}
            \def\name{Tab.\hspace{10pt}}
        \fi
    \fi
    \setcounter{anlagen}{\type}

    % only here because the \type-counter is one lower (later it will count up like normal)
    % -> reason... it's called too early but can't called later because of dependencies other types
    % works only in the last section of the paper so it should be fine :)
    \refstepcounter{anlagen}

    \addcontentsline{alt}{anlagen}
    {\name\protect\numberline{\theanlagen}\quad#1}\par
}

\newenvironment{longfigure}{\captionsetup{type=figure}}{}

% AUTO REMOVE/INSERT Anlagenverzeichnis
\newtotcounter{anlagenentries}  % stepCounter within table and figure to check if used
\newcommand{\renewFigTabCap} {
    % \caption ruft zusätzlich \addItemToAnlageverzeichnis auf
    \let\oldCap=\caption
    \renewcommand{\caption}[1]{\addItemToAnlageverzeichnis{##1}\oldCap{##1}}

    % redefine table and figure -> table and figure set a global variable on the specific value
    \let\oldTab=\table
    \renewcommand{\table}{\def\typeOfCap{tab}\stepcounter{anlagenentries}\oldTab}

    \let\oldFig=\figure
    \renewcommand{\figure}{\def\typeOfCap{fig}\stepcounter{anlagenentries}\oldFig}

    \let\oldLongFig=\longfigure
    \renewcommand{\longfigure}{\def\typeOfCap{fig}\stepcounter{anlagenentries}\oldLongFig}
}


% AUTO REMOVE/INSERT Literaturverzeichnis
\newcounter{totalbibentries}
\newcommand*{\listcounted}{}

\makeatletter
\AtDataInput{
    \xifinlist{\abx@field@entrykey}\listcounted
    {}
    {\stepcounter{totalbibentries}
        \listxadd\listcounted{\abx@field@entrykey}}
}
\makeatother


% ACRO SETUP
\acsetup{
    list/heading = section*,
    list/name = {Abkürzungsverzeichnis},
    list/template = description,
    make-links = true,
    link-only-first = false
}

% Standard Abkürzungsverzeichnis überschreiben -> einheitliche Einrückung
\RenewAcroTemplate[list]{description}{%
    \acronymsmapT{%
        \AcroAddRow{%
            \textbf{\acrowrite{short}}%
            &
            \acrowrite{long}%
            \acropages
            {\acrotranslate{page}\nobreakspace}%
            {\acrotranslate{pages}\nobreakspace}%
            \vspace{10pt}
            \tabularnewline
        }%
    }%
    \acroheading
    \acropreamble
    \noindent
    \begin{tabular}{@{}ll}
        \AcronymTable
    \end{tabular}
}

% AUTO REMOVE/INSERT Abkürzungsverzeichnis
% Abkürzungsverzeichnis überschreibt \UseAcroTemplate für \ac
% New Counter to count used acronyms:
\newtotcounter{acro_num}
\def\oldUseAcroTemplate{} \let\oldUseAcroTemplate=\UseAcroTemplate
\def\UseAcroTemplate{\stepcounter{acro_num}\oldUseAcroTemplate}


% INDENTION SETUP
% Abstände und Einrückungen abhängig von config.tex ein-/ausschalten
\if\CEINR 0
    \setlength{\parskip}{6pt}
    \setlength{\parindent}{0cm}
\fi

% ab hier ist das Arbeitsverzeichnis das root Verzeichnis (gilt primär für \input)

% Abkürzungen müssen früher eingefügt werden, da sie
% nicht als Text, sondern als Variablen definiert werden
% Definieren Sie hier Ihre Abkürzungen und Glossar-Einträge anhand der Beispiele.
% Wenn Sie diese dann im Text verwenden, rufen Sie einfach \gls{key} auf, z.B. \gls{ac:dhge}.
% LaTeX kümmert sich um den Rest.
% Alle Abkürzungen auch ohne Verweis darauf generieren ist in der README.md dokumentiert.

% Eine ausführliche, anfängerfreundliche Dokumentation ist unter https://www.overleaf.com/learn/latex/Glossaries abrufbar.

\newglossaryentry{gls:gloss}{
    name={Glossar},
    description={Ein Glossar ist eine Liste von Wörtern mit beigefügten Bedeutungserklärungen oder Übersetzungen. (Wikipedia)}
}

\newacronym[]{ac:dhge}{DHGE}{Duale Hochschule Gera-Eisenach}



% Konfiguration globaler Definitionen

% PDF Metadata
\hypersetup{
    pdftitle={\CTITLE},
    pdfauthor={\CAUTHOR}
}

\addbibresource	{../literatur.bib}

% Hier beginnt der Spaß
% setzt variablen für den Titel -> \maketitle in deckblatt.tex
\title{{\LARGE \textbf{\CTITLE}}}
\author{}
\date{}


\begin{document}
\pagenumbering{gobble}
    % lade Deckblatt ohne Nummerierung
    % DEFINITION SECTION

% legt den hSpace fest um die Einträge mittig zu platzieren
% 0.4375 berechnet sich aus:
% ((textwidth / 2) - (margin_left - margin_right)) / textwidth
% mit textwidth = pagewidth - margin_left - margin_right
% Dann muss nur noch parindent abgezogen werden
\def\defaultHSpace{\hspace{-\parindent}\hspace{0.4375\textwidth}}

\newcommand{\markBox}[2]
{
    \ifnum#1 = 1
        \def\checkboxes{#2 {$\boxtimes$} #2 {$\square$} #2 {$\square$} #2 {$\square$}}
    \else\ifnum#1 = 2
        \def\checkboxes{#2 {$\square$} #2 {$\boxtimes$} #2 {$\square$} #2 {$\square$}}
    \else\ifnum#1 = 3
        \def\checkboxes{#2 {$\square$} #2 {$\square$} #2 {$\boxtimes$} #2 {$\square$}}
    \else\ifnum#1 = 4
        \def\checkboxes{#2 {$\square$} #2 {$\square$} #2 {$\square$} #2 {$\boxtimes$}}
    \else
        \def\checkboxes{#2 {$\square$} #2 {$\square$} #2 {$\square$} #2 {$\square$}}
    \fi\fi\fi\fi

    \hspace*{-.5cm}\checkboxes
}

% Definition der Deckblatt-Einträge

\newcommand{\deckblattEntry}[2] {
    \begin{tabular}{rl}
        \defaultHSpace{} & \\ #1: & #2
    \end{tabular}
    % folgende newline ist notwendig damit die Formatierung angewendet wird

}


% DECKBLATT STRUKTUR SECTION
\vspace{\fill}
\maketitle

\if\CARBEIT B
    \begin{center}
        {\LARGE\bf Bachelorarbeit}

        \vspace{0.5cm}vorgelegt am \CDATUM
    \end{center}

    \vspace{1cm}
\else
    \begin{tabular}{rcccc}
        \defaultHSpace{} & I & II & III & IV \\
        {Projektarbeit Nr.}  \markBox{\CARBEIT}{&}
    \end{tabular}

    \deckblattEntry{vorgelegt am}{\CDATUM}
\fi

\deckblattEntry{von}{\CAUTHOR}
\deckblattEntry{Matrikelnummer}{\CMATRIKEL}
\deckblattEntry{DHGE Campus}{\CCAMPUS}
\deckblattEntry{Studienbereich}{\CBEREICH}
\deckblattEntry{Studiengang}{\CSTUDIENGANG}
\deckblattEntry{Kurs}{\CKURS}
\deckblattEntry{Ausbildungsstätte}{\CBETRIEB}
\deckblattEntry{\BETREUER}{\CBETREUER}

\vspace*{\fill}

\pagebreak


    % Lade Sperrvermerk abhängig von der Einstellung in config.tex
    \if\CSPERRVERMERK 1
        \def\CARBEITTYPNAME{Projektarbeit}

\if\CARBEIT B
    \def\CARBEITTYPNAME{Bachelorarbeit}
\fi

\vspace*{5.5cm}
\begin{center}
    {\LARGE\bf Sperrvermerk}

    \vspace*{1cm}
    Die vorgelegte \CARBEITTYPNAME{} mit dem Titel \enquote{\CTITLE{}} basiert auf internen, vertraulichen Daten und Informationen des Unternehmens \CBETRIEB{}.

    Diese \CARBEITTYPNAME{} darf nur vom Erst- und Zweitgutachter sowie berechtigten Mitgliedern des Prüfungsausschusses eingesehen werden.
    Eine Vervielfältigung und Veröffentlichung der \CARBEITTYPNAME{} ist auch auszugsweise nicht erlaubt.

    Die Vervielfältigung und Veröffentlichung der \CARBEITTYPNAME{} sowie die Einsichtnahme durch Dritte bedarf der ausdrücklichen
    Zustimmung des Verfassers und des Unternehmens.
\end{center}
\cleardoublepage

    \fi

    % Lade Thesenblatt abhängig von der Einstellung in config.tex
    \if\CARBEIT B
        % auskommentieren was nicht genutzt wird

\baFormat{Thesen zur Bachelorarbeit}
{
    \begin{enumerate}
        \item These 1
            \vspace{0.5cm}

        \item These 2
            \vspace{0.5cm}

        \item These 3
            \vspace{0.5cm}

        \item These 4
            \vspace{0.5cm}

        \item These 5
            \vspace{0.5cm}
    \end{enumerate}
}

\baFormat{Autorreferat zur Bachelorarbeit}{
    Beispieltext
}

    \fi

    % Lade Abstract abhängig von der Einstellung in config.tex
    \if\CHASABSTRACT 1
        \section*{Abstract}
        Franz jagt im komplett verwahrlosten Taxi quer durch Bayern.
Falsches Üben von Xylophonmusik quält jeden größeren Zwerg.

Es ist Mittwoch, meine Kerle. Oder ist es Montag, meine Mümmler?

Lorem ipsum dolor sit amet.

        \newpage % Sections haben hier noch keinen automatischen Seitenumbruch
    \fi

    % INHALTSVERZEICHNIS
    \pagenumbering{Roman} \setcounter{page}{1}
    \tableofcontents{\fancyfoot{}}
    \if\CKUSCHE 1 % im Kusche Mode erscheinen die Anlagen im InhaltsVZ
        \phantomsection
        \listofanlagen
    \fi
    \cleardoublepage

    % ABBILDUNGSVERZEICHNIS
    \phantomsection
    \conditionalLoF

    % TABELLENVERZEICHNIS
    \phantomsection
    \conditionalLoT

    % ABKÜRZUNGSVERZEICHNIS
    % wird nur dann generiert, wenn mindestens ein mal \ac oder ein verwandter Befehl aufgerufen wurde und damit nicht leer ist
    \ifnum\totvalue{acro_num}>0
        \printacronyms
        \addcontentsline{toc}{section}{Abkürzungsverzeichnis}
        \cleardoublepage
    \fi

    \setcounter{savepage}{\arabic{page}}

    % MAIN CONTENT
    \pagenumbering{arabic}
    \section{Zitat und Abkürzung Test Section}

Diese Arbeit ist ein Test welcher f\"ur das \ac{dhge} LaTeX Template vorgesehen ist.\footcite{Xmisc}

Das Template unterstützt auch einen weiteren cite Befehl welcher platzsparender ist.\supercite{Xmisc}

\cleardoublepage

\subsection{Command Test Subsection}

\doubleunderline{$150\mathrm{\Omega}$}

\subsection{Bilder Test Subsection}

\dhgefigure[h]{img}{scale=0.25}{Ein Testbild}{fig:test}[Xmisc][S. 17ff]
\dhgefigure[h]{img}{scale=.6}{Ein Testbild}{fig:test2}

    \cleardoublepage

    % LITERATURVERZEICHNIS
    % TODO Formatierung

    % Im Kusche-Mode wird arabische Nummerierung beibehalten
    \if\CKUSCHE 0
        \pagenumbering{Roman} \setcounter{page}{\thesavepage}
    \fi

    % wenn keine Literatur verwendet (zitiert) wird, erstelle kein Literaturverzeichnis
    % im Kusche-Mode wird das LiteraturVZ später generiert
    \if\CKUSCHE 0
        \ifnum\thetotalbibentries>0
            \printbibliography[title=Literaturverzeichnis]
            \addcontentsline{toc}{section}{Literaturverzeichnis}
            \cleardoublepage
        \fi
    \fi


    % TODO %ANLAGENVERZEICHNIS UND ANLAGEN
    % erzeugt ein Anlagenkapitel und fügt es zum InhaltsVZ hinzu, insofern Anlagen existieren.

    % \listofanlagen generiert ein Anlagenverzeichnis
    % im Kusche-Modus wird kein Anlagenverzeichnis erzeugt, aber Anlagen im Inhaltsverzeichnis geführt
    \ifnum\totvalue{anlagenentries}>0
    \section*{Anlagen}
    \addcontentsline{toc}{section}{Anlagen}
    \if\CKUSCHE 0
        \begin{spacing}{2}
            \listofanlagen
        \end{spacing}
        \cleardoublepage
    \fi
    \fi

    \if\CKUSCHE 1
        % die Kapitel.lfd-Nummerierung wird im Kusche Mode ab hier wieder deaktiviert,
        % da Anhänge mit Buchstaben nummeriert werden sollen (TODO: das funktioniert noch nicht, help wanted)
        \counterwithout{figure}{section}
        \counterwithout{table}{section}

        % deaktiviere Kapitel in Fußzeile im Kusche Mode ab hier
        \fancyfoot      [L]{}
    \fi

    \renewFigTabCap % Verhalten von Figure und Table Environments sowie \caption wird verändert (TODO: wie genau? LG, ZPM)
    \begin{table}[H]
    \caption{Test Table}
    \begin{tabular}{| l | l |}
        Test & Test\\
        \hline Test & test
    \end{tabular}
\end{table}

\dhgefigure[h]{img}{scale=0.25}{Ein Testbild}{fig:anlagentest}[Xmisc][S. 17ff]

\begin{figure}[H]
    \centering
    \caption{test}
    \includegraphics[scale=0.75]{img}
    \label{fig:anlagentest2}
\end{figure}

    \cleardoublepage

    % Im Kusche-Mode kommt das LiteraturVZ zuletzt
    \if\CKUSCHE 1
        \ifnum\thetotalbibentries>0
            \printbibliography[title=Literaturverzeichnis]
            \addcontentsline{toc}{section}{Literaturverzeichnis}
            \cleardoublepage
        \fi
    \fi

    % EHRENWÖRTLICHE ERKLÄRUNG
    \pagestyle{empty}
    \pagenumbering{gobble}
    \section*{Ehrenwörtliche Erklärung}
    \addcontentsline{toc}{section}{Ehrenwörtliche Erklärung}
    Ich erkläre hiermit ehrenwörtlich,
\begin{flushleft}
\begin{enumerate}[leftmargin=0.5cm]
    \item 	dass ich meine {\if\CARBEIT BBachelorarbeit\else Projektarbeit\fi}
    mit dem Thema:  \\
    \vspace*{1cm}
            \textbf{\CTITLE} \\
    \vspace*{1cm}
            ohne fremde Hilfe angefertigt habe, \\
    \item	dass ich die Übernahme wörtlicher Zitate aus der Literatur sowie die Verwendung der
            Gedanken anderer Autoren an den entsprechenden Stellen innerhalb der Arbeit gekennzeichnet habe und  \\
    \item	dass ich meine {\if\CARBEIT BBachelorarbeit\else Projektarbeit\fi}
    bei keiner anderen Prüfung vorgelegt habe. \\
    \vspace*{1cm}
\end{enumerate}
\noindent
Ich bin mir bewusst, dass eine falsche Erklärung rechtliche Folgen haben wird.
\end{flushleft}
\vspace*{\fill}
\begin{tabular} {lrl}
    \hspace{5.5cm} & \hspace{3cm} & \hspace{5.5cm} \\
    \hrulefill & & \hrulefill \\
    Ort, Datum & & Unterschrift
\end{tabular}
\vspace*{\fill}

\cleardoublepage


    \if\CARBEIT B
        \input{components/freigabe.tex}
    \fi

\end{document}
