% ROOT & PACKAGE SETUP
\documentclass	[a4paper, 12pt]{article}

% preamble input
% file to include additional commands, packages or anything else you'd like to include in the preamble


% management for all packages
\usepackage		[a4paper,
    	         inner  = 3cm,
                 outer  = 2cm,
                 top    = 2.5cm,
                 bottom = 2.5cm]{geometry}
\usepackage		{setspace}
\usepackage		[hyperfootnotes = false,
                 hidelinks]{hyperref}
\usepackage		{amssymb}
\usepackage		{fancyhdr}
\usepackage		[version = 3]{acro}
\usepackage		{enumitem}
\usepackage		[style=german]{csquotes}
\usepackage		[backend=biber,
                 style        = alphabetic,
                 citestyle    = components/alphabetic-ibid,
                 giveninits   = true,
                 ibidtracker  = true,
                 minbibnames  = 3,
                 minalphanames= 3]{biblatex}
\usepackage		[ngerman]{babel}
\usepackage		{footmisc}
\usepackage		{graphicx}
\usepackage		{caption}
\usepackage		{xparse}
\usepackage		{float}
\usepackage		{tocloft}
\usepackage     {lmodern}
\usepackage     {totcount}
\usepackage     {chngcntr}

\if\CFANCYFONTS 1
\usepackage     [scaled=0.88]{beraserif} % Bera Serifen Font
\usepackage     [scaled=0.85]{berasans} % Bera Sans Font
\usepackage     [scaled=0.84]{beramono} % Bera Mono Font
\usepackage     [T1]{fontenc}
\usepackage     {mathpazo} % Palatino Font
\usepackage     [T1,small,euler-digits]{eulervm} % Euler Font
\usepackage     {listings}
\lstset{
	basicstyle=\ttfamily,
    breaklines=true
}
\fi

% VERALTETE PAKETE
% nicht mehr benötigt, aber für Rückwärtskompatibilität noch enthalten
% einkommentieren, wenn nach einem Template-Update Probleme auftauchen

%\usepackage		[utf8]{inputenc}
%\usepackage		{titletoc}
%\usepackage		{csquotes,xpatch}
%\usepackage		{ifthen}
%\usepackage		{etoolbox}
%\usepackage     {icomma}


% setups and commands
% dhgefigure -> ...
\DeclareDocumentCommand{\dhgefigure}{O{tbp} m m m m O{} O{}}
{
	\begin{figure}[#1]
		\begin{center}
			\includegraphics[#3]{#2}
		\end{center}
		\caption{#4}
		\label{#5}

		\ifx #6\empty \else
			\ifx #7\empty \else
				{\small \protect \textbf{Quelle:} \cite[#7]{#6}}
			\fi
		\fi

	\end{figure}
}

\newcommand{\markBox}[2]
{
	\def\default{#2 {$\square$} #2 {$\square$} #2 {$\square$} #2 {$\square$}}

	\ifnum#1 = 1
		\def\default{#2 {$\boxtimes$} #2 {$\square$} #2 {$\square$} #2 {$\square$}}
	\else
		\ifnum#1 = 2
			\def\default{#2 {$\square$} #2 {$\boxtimes$} #2 {$\square$} #2 {$\square$}}
		\else
			\ifnum#1 = 3
				\def\default{#2 {$\square$} #2 {$\square$} #2 {$\boxtimes$} #2 {$\square$}}
			\else
				\ifnum#1 = 4
					\def\default{#2 {$\square$} #2 {$\square$} #2 {$\square$} #2 {$\boxtimes$}}
				\fi
			\fi
		\fi
	\fi
	\hspace*{-.5cm}\default
}

% Doppelte Unterstreichung
\newcommand{\doubleunderline}[1]{
	\underline{\underline{#1}}
}

% definiert eine neue Liste für das Anlagenverzeichnis
\newcommand{\listexamplename}{\vspace*{-20pt}}
\newlistof{anlagen}{alt}{\listexamplename}

% Befehl welcher ein Item dem Anlagenverzeichnis hinzufügt
\newcommand{\ATA}[1]{%
	\def\fig{fig}
	\def\tab{tab}

	\ifx\fig\typeOfCap
		\def\type{\thefigure}
		\def\name{Abb.\hspace{8pt}}
	\else \ifx\tab\typeOfCap
			\def\type{\thetable}
			\def\name{Tab.\hspace{10pt}}
		\fi
	\fi
	\setcounter{anlagen}{\type}

	% only here because the \type-counter is one lower (later it will count up like normal)
	% -> reason... it's called too early but can't called later because of dependencies other types
	% works only in the last section of the paper so it should be fine :)
	\refstepcounter{anlagen}

	\addcontentsline{alt}{anlagen}
	{\name\protect\numberline{\theanlagen}\quad#1}\par
}

\newenvironment{longfigure}{\captionsetup{type=figure}}{}

\newtotcounter{anlagenentries}
\newcommand{\renewFigTabCap} {
	% caption ruft jetzt \ATA{} auf, welches das jeweilige Objekt zum "Anlageverzeichnis" hinzufügt
	\let\oldCap=\caption
	\renewcommand{\caption}[1]{\ATA{##1}\oldCap{##1}}

	% redefine table and figure -> table and figure set a global variable on the specific value
	\let\oldTab=\table
	\renewcommand{\table}{\def\typeOfCap{tab}\stepcounter{anlagenentries}\oldTab}

	\let\oldFig=\figure
	\renewcommand{\figure}{\def\typeOfCap{fig}\stepcounter{anlagenentries}\oldFig}

	\let\oldLongFig=\longfigure
	\renewcommand{\longfigure}{\def\typeOfCap{fig}\stepcounter{anlagenentries}\oldLongFig}
}

% Formatierung der Bachelorarbeit: Autorreferat und Thesenblatt
\newcommand{\baFormat}[2]{
	\begin{center}
		{\LARGE\bf #1}

		\vspace{0.7cm}
		{\large\bf\enquote{\CTITLE}}

		\vspace{0.5cm}
		von \CAUTHOR
	\end{center}

	\vspace{1.5cm}

	{#2}

	\cleardoublepage
}

% CONSTANTS

    % Projektarbeit Nr. (1 bis 4) oder Bachelorarbeit (B)
    \def\CARBEIT      {B}

    % Title der Arbeit
    \def\CTITLE		    {THEMA}

    % Author der Arbeit
    \def\CAUTHOR		{AUTHOR}

    % "vorlege am" - Datum
    \def\CDATUM		    {\today}

    % Martikelnummer des Authors
    \def\CMATRIKEL		{MATRIKELNUMMER}

    % Kurs des Auhtors
    \def\CKURS		    {KURS}

    % DHGE Campus des Authors (Gera/Eisenach)
    \def\CCAMPUS		{Gera}

    % Studienbereich des Authors
    \def\CBEREICH		{Technik}

    % Studiengang des Authors
    \def\CSTUDIENGANG	{STUDIENGANG}

    % Betrieb des Authors (nur der Name des Betriebs keine Adresse)
    \def\CBETRIEB		{FIRMA}

    % Betreuer der Arbeit
    \def\CBETREUER		{BETREUER}

% FONT SETUP
% schönere Fonts, aber optional. Zum deaktivieren CFANCYFONTS in config.tex auf 0 setzen
\if\CFANCYFONTS 1
    \lstset{
        basicstyle=\ttfamily,
        breaklines=true
    }
\fi


% VARIABLE SETUP
% die veraltete CAUTHOR Variable wird automatisch befüllt
\def\CAUTHOR{\CAUTHORVOR\ \CAUTHORNACH}

\if\CARBEIT B
    \def\BETREUER{Gutachter}
\else
    \def\BETREUER{Betreuer}
\fi


% DOCUMENT SETUP
\onehalfspacing % 1.5 line spacing
% TODO: sollte theoretisch keine Verwendung mehr haben
% \widowpenalty10000
% \clubpenalty10000


% INHALTSVERZEICHNIS SETUP
\renewcommand{\contentsname}{Inhaltsverzeichnis}
\cftsetindents{section}{0em}{4em}
\cftsetindents{subsection}{0em}{4em}
\cftsetindents{subsubsection}{0em}{4em}
\setcounter{tocdepth}{3}
\setcounter{secnumdepth}{5}


% ABBILDUNGEN UND TABELLEN SETUP
\renewcommand{\listfigurename}{Abbildungsverzeichnis}
\renewcommand{\listtablename}{Tabellenverzeichnis}

\addto{\captionsngerman}{
    \renewcommand*{\figurename}{Abb.}
    \renewcommand*{\tablename}{Tab.}
}

\addtocontents{lof}{\linespread{2}\selectfont}
\addtocontents{lot}{\linespread{2}\selectfont}

\makeatletter
\renewcommand{\cftfigpresnum}{Abb. }
\renewcommand{\cfttabpresnum}{Tab. }

\setlength{\cftfignumwidth}{2cm}
\setlength{\cfttabnumwidth}{2cm}

\setlength{\cftfigindent}{0cm}
\setlength{\cfttabindent}{0cm}
\makeatother

% MODUS KUSCHE: ABBILDUNGEN UND TABELLEN SETUP
% im Kusche-Mode sollen Abbildungen nach Kapitel.lfd nummeriert werden
\if\CKUSCHE 1
    \counterwithin{figure}{section}
    \counterwithin{table}{section}
\fi


% CAPTION SETUP
\captionsetup{
    font = small,
    labelfont = bf,
    singlelinecheck = false,
    skip = 10pt,
    belowskip = 0pt
}


% CITATION SETUP
\renewcommand*{\labelalphaothers}{\textsuperscript{}}


% HEADERS & FOOTERS
\pagestyle		{fancyplain}
\fancyhf		{}
\renewcommand	{\headrulewidth}{0pt}
\renewcommand	{\footrulewidth}{0pt}
\setlength		{\headheight}{15pt}

% KUSCHE MODE: HEADERS & FOOTERS

\fancyfoot      [R]{\thepage} % nach den neuen Anforderungen sind Seitenzahlen unten rechts, geht d'accord mit dem Kusche-Mode
\if\CKUSCHE 1
    \fancyfoot      [L]{\leftmark} % im Kusche-Mode erscheint linksbündig das Kapitel in der Fußzeile
\fi


% FOOTNOTE SETUP
\renewcommand{\footnotelayout}{\hspace{0.5em}}

% COUNTER
% Zweck: in römischen Zahlen weiter zählen, nachdem der Counter von arabisch zurück geändert wird
\newcounter{savepage}


% SECTION SETUP
% sections sollen mit Seitenumbruch beginnen

\let\stdsection\section
\renewcommand\section{\newpage\stdsection}


% MATHRM ADJUSTMENTS
% Überschreiben von \mathrm{} -> einheitlichen Abstand einfügen
\let\oldMathrm\mathrm
\renewcommand{\mathrm}[1]{\,\oldMathrm{#1}}

% PATH SETUP
% root ist ist das neue Arbeitsverzeichnis
\makeatletter
\def\input@path{{../}{path1/}}
\makeatother

\graphicspath	{{../assets/img/}}


% AUTO REMOVE/INSERT ABBILDUNGSVERZEICHNIS & TABELLENVERZEICHNIS
% Conditionals um AbbildungVZ und TabellenVZ nur zu rendern, wenn sie nicht leer sind
\newtotcounter{figCount}
\newtotcounter{tabCount}
\let\oldTabTOC=\table
\let\oldFigTOC=\figure
\renewcommand{\figure}{\stepcounter{figCount}\oldFigTOC}
\renewcommand{\table}{\stepcounter{tabCount}\oldTabTOC}

\newcommand{\conditionalLoF}{
    \ifnum\totvalue{figCount}>0
        \addcontentsline{toc}{section}{\listfigurename}
        \listoffigures
        \cleardoublepage
    \fi
}
\newcommand{\conditionalLoT}{
    \ifnum\totvalue{tabCount}>0
        \addcontentsline{toc}{section}{\listtablename}
        \listoftables
        \cleardoublepage
    \fi
}


% ANLAGENVERZEICHNIS SETUP
% definiert eine neue Liste für das Anlagenverzeichnis
\newcommand{\listanlageverzeichnis}{\vspace*{-20pt}}
\newlistof{anlagen}{alt}{\listanlageverzeichnis}

% Befehl welcher ein Item dem Anlagenverzeichnis hinzufügt
\newcommand{\addItemToAnlageverzeichnis}[1]{%
    \def\fig{fig}
    \def\tab{tab}

    \ifx\fig\typeOfCap
        \def\type{\thefigure}
        \def\name{Abb.\hspace{8pt}}
    \else \ifx\tab\typeOfCap
            \def\type{\thetable}
            \def\name{Tab.\hspace{10pt}}
        \fi
    \fi
    \setcounter{anlagen}{\type}

    % only here because the \type-counter is one lower (later it will count up like normal)
    % -> reason... it's called too early but can't called later because of dependencies other types
    % works only in the last section of the paper so it should be fine :)
    \refstepcounter{anlagen}

    \addcontentsline{alt}{anlagen}
    {\name\protect\numberline{\theanlagen}\quad#1}\par
}

\newenvironment{longfigure}{\captionsetup{type=figure}}{}

% AUTO REMOVE/INSERT Anlagenverzeichnis
\newtotcounter{anlagenentries}  % stepCounter within table and figure to check if used
\newcommand{\renewFigTabCap} {
    % \caption ruft zusätzlich \addItemToAnlageverzeichnis auf
    \let\oldCap=\caption
    \renewcommand{\caption}[1]{\addItemToAnlageverzeichnis{##1}\oldCap{##1}}

    % redefine table and figure -> table and figure set a global variable on the specific value
    \let\oldTab=\table
    \renewcommand{\table}{\def\typeOfCap{tab}\stepcounter{anlagenentries}\oldTab}

    \let\oldFig=\figure
    \renewcommand{\figure}{\def\typeOfCap{fig}\stepcounter{anlagenentries}\oldFig}

    \let\oldLongFig=\longfigure
    \renewcommand{\longfigure}{\def\typeOfCap{fig}\stepcounter{anlagenentries}\oldLongFig}
}


% AUTO REMOVE/INSERT Literaturverzeichnis
\newcounter{totalbibentries}
\newcommand*{\listcounted}{}

\makeatletter
\AtDataInput{
    \xifinlist{\abx@field@entrykey}\listcounted
    {}
    {\stepcounter{totalbibentries}
        \listxadd\listcounted{\abx@field@entrykey}}
}
\makeatother


% Glossar Setup
\makeglossaries

% INDENTION SETUP
% Abstände und Einrückungen abhängig von config.tex ein-/ausschalten
\if\CEINR 0
    \setlength{\parskip}{6pt}
    \setlength{\parindent}{0cm}
\fi
  % -> default dir is set to "../" in this file
% Definieren Sie hier Ihre Abkürzungen und Glossar-Einträge anhand der Beispiele.
% Wenn Sie diese dann im Text verwenden, rufen Sie einfach \gls{key} auf, z.B. \gls{ac:dhge}.
% LaTeX kümmert sich um den Rest.
% Wenn alle Abkürzungen auch ohne Verweis darauf generiert werden sollen, ist ein Schalter dafür in config.tex verfügbar.
% Eine ausführliche, anfängerfreundliche Dokumentation ist unter https://www.overleaf.com/learn/latex/Glossaries abrufbar.

\newglossaryentry{gls:gloss}{
    name={Glossar},
    description={Ein Glossar ist eine Liste von Wörtern mit beigefügten Bedeutungserklärungen oder Übersetzungen. (Wikipedia)}
}

\newacronym[]{ac:dhge}{DHGE}{Duale Hochschule Gera-Eisenach}



% configuration of global definitions

% PDF Metadata
\hypersetup{
	pdftitle={\CTITLE},
	pdfauthor={\CAUTHOR}
}

\addbibresource	{../literatur.bib}
\graphicspath	{{../assets/img/}}

% opening
\title{{\LARGE \textbf{\CTITLE}}}
\author			{}
\date			{}

\begin{document}
\pagenumbering	{gobble}

\vspace{\fill}
\maketitle

\if\CARBEIT B

	\begin{center}
		{\LARGE\bf Bachelorarbeit}
		
		\vspace{0.5cm}vorgelegt am \CDATUM
	\end{center}

	\vspace{1cm}

	\def\BETREUER{Gutachter}

\else

	\begin{tabular}{rcccc}
		\hspace{0.45\textwidth} &       I       &     II      &     III     &     IV      \\
	{Projektarbeit Nr.}  \markBox{\CARBEIT}{&}
	\end{tabular}

	\begin{tabular}{rl}
		\hspace{0.45\textwidth} &       \\
	   vorgelegt am: & \CDATUM
   \end{tabular}

   \def\BETREUER{Betreuer}

\fi

\begin{tabular}{rl}
	\hspace{0.45\textwidth} &              \\
	        von: & \CAUTHOR
\end{tabular}

\begin{tabular}{rl}
	\hspace{0.45\textwidth} &         \\
	 Matrikelnummer: & \CMATRIKEL
\end{tabular}

\begin{tabular}{rl}
	\hspace{0.45\textwidth} &      \\
	DHGE Campus: & \CCAMPUS
\end{tabular}

\begin{tabular}{rl}
	 \hspace{0.45\textwidth} &         \\
	Studienbereich: & \CBEREICH
\end{tabular}

\begin{tabular}{rl}
	\hspace{0.45\textwidth} &                       \\
	Studiengang: & \CSTUDIENGANG
\end{tabular}

\begin{tabular}{rl}
	\hspace{0.45\textwidth} &       \\
	       Kurs: & \CKURS
\end{tabular}

\begin{tabular}{rl}
	\hspace{0.45\textwidth} &          \\
	Ausbildungsstätte: & \CBETRIEB
\end{tabular}

\begin{tabular}{rl}
	\hspace{0.45\textwidth} &          \\
	\BETREUER: & \CBETREUER
\end{tabular}

\vspace*{\fill}

\pagebreak



\if\CSPERRVERMERK 1
	\def\CARBEITTYPNAME{Projektarbeit}

\if\CARBEIT B
    \def\CARBEITTYPNAME{Bachelorarbeit}
\fi

\vspace*{5.5cm}
\begin{center}
    {\LARGE\bf Sperrvermerk} 
    
    \vspace*{1cm}
    Die vorgelegte \CARBEITTYPNAME{} mit dem Titel \enquote{\CTITLE{}} basiert auf internen, vertraulichen Daten und Informationen des Unternehmens \CBETRIEB{}.

    Diese \CARBEITTYPNAME{} darf nur vom Erst- und Zweitgutachter sowie berechtigten Mitgliedern des Pr\"ufungsausschusses eingesehen werden. 
    Eine Vervielf\"altigung und Ver\"offentlichung der Bachelorarbeit ist auch auszugsweise nicht erlaubt.

    Die Vervielf\"altigung und Ver\"offentlichung der \CARBEITTYPNAME{} sowie die Einsichtnahme durch Dritte bedarf der ausdrücklichen 
    Zustimmung des Verfassers und des Unternehmens.
\end{center}
\cleardoublepage

\fi


\if\CARBEIT B
	% auskommentieren was nicht genutzt wird

\baFormat{Thesen zur Bachelorarbeit}
{
    \begin{enumerate}
        \item These 1
        \vspace{0.5cm}

        \item These 2
        \vspace{0.5cm}

        \item These 3
        \vspace{0.5cm}

        \item These 4
        \vspace{0.5cm}

        \item These 5
        \vspace{0.5cm}
    \end{enumerate}
}

\baFormat{Autorreferat zur Bachelorarbeit}{
    Beispieltext
}
\fi

\if\HASABSTRACT 1
    \section*{Abstract}
% hier können Sie Ihr Abstract schreiben

    \newpage % for some reason, the section newpage override doesn't work here, so just do it manually
\fi

% INHALTSVERZEICHNIS
\pagenumbering{Roman} \setcounter{page}{1}
\tableofcontents{\fancyfoot{}}
\cleardoublepage

% ABBILDUNGSVERZEICHNIS
\phantomsection
\conditionalLoF

% TABELLENVERZEICHNIS
\phantomsection
\conditionalLoT

% ABKÜRZUNGSVERZEICHNIS
% will only be created if \ac or similar used at least once
\ifnum\totvalue{acro_num}>0
	\printacronyms
	\addcontentsline{toc}{section}{Abkürzungsverzeichnis}
	\cleardoublepage
\fi


\setcounter{savepage}{\arabic{page}}

% MAIN CONTENT
\pagenumbering	{arabic}
% hier können Sie Ihre Arbeit schreiben.
% für ein Beispiel siehe `build/tests/`
\newglossaryentry{gls-id}{name={name},description={desc}}

\cleardoublepage

% LITERATURVERZEICHNIS
% TODO Formatierung
% Im Kusche-Mode wird arabische Nummerierung beibehalten
\if\KUSCHE 0
\pagenumbering	{Roman} \setcounter{page}{\thesavepage}
\fi

% wenn keine Literatur verwendet (zitiert) wird, erstelle kein Literaturverzeichnis
% im Kusche-Mode wird das LiteraturVZ später generiert
\if\KUSCHE 0
    \ifnum\thetotalbibentries>0
        \printbibliography[title=Literaturverzeichnis]
        \addcontentsline{toc}{section}{Literaturverzeichnis}
        \cleardoublepage
    \fi
\fi


% TODO %ANLAGENVERZEICHNIS UND ANLAGEN
% im Kusche-Modus wird kein Anlagenverzeichnis erzeugt, aber Anlagen im Inhaltsverzeichnis geführt (TBD)
\if\KUSCHE 0
    \ifnum\totvalue{anlagenentries}>0
        \section*{Anlagen}
        \begin{spacing}{2}
            \listofanlagen
        \end{spacing}
        \addcontentsline{toc}{section}{Anlagen}
        \cleardoublepage
    \fi
\fi

\renewFigTabCap % change the behavior from the env figure and table and the command \caption{}
% Beispiel einer Tabelle in den Anlagen
\begin{table}[H]
    \begin{tabular}{c c}
        1 & 1\\
        1 & 1\\  
    \end{tabular}
    \caption{Test Beispiel für eine einfache Tabelle \label{tab:testBsp}}
\end{table}
% Befehl welcher es dem Verzeichnis hinzufügt
\addtoanlagen{tab}{testBsp}

\cleardoublepage


% Im Kusche-Mode kommt das LiteraturVZ zuletzt
\if\KUSCHE 1
    \ifnum\thetotalbibentries>0
        \printbibliography[title=Literaturverzeichnis]
        \addcontentsline{toc}{section}{Literaturverzeichnis}
        \cleardoublepage
    \fi
\fi

% EHRENWÖRTLICHE ERKLÄRUNG
\pagestyle{empty}
\pagenumbering{gobble}
\section*{Ehrenwörtliche Erklärung}
\addcontentsline{toc}{section}{Ehrenwörtliche Erklärung}
Ich erkläre hiermit ehrenwörtlich,
\begin{enumerate}[leftmargin=0.5cm]
	\item 	dass ich meine Projektarbeit mit dem Thema:  \\
			\textbf{\CTITLE} \\
			ohne fremde Hilfe angefertigt habe, \\
	\item	dass ich die Übernahme wörtlicher Zitate aus der Literatur sowie die Verwendung der
			Gedanken anderer Autoren an den entsprechenden Stellen innerhalb der Arbeit gekennzeichnet habe und  \\
	\item	dass ich meine Projektarbeit/Studienarbeit/Bachelorarbeit bei keiner anderen Prüfung vorgelegt habe. \\\\
			Ich bin mir bewusst, dass eine falsche Erklärung rechtliche Folgen haben wird. \\\\
\end{enumerate}
\vspace*{\fill}
\begin{tabular} {lrl}
	\hspace{6cm} & \hspace{3cm} & \hspace{6cm} \\
	\hrulefill & & \hrulefill \\
	Ort, Datum & & Unterschrift
\end{tabular}
\vspace*{\fill}


\if\CARBEIT B
	\begin{center}
    {\LARGE\bf Freigabeerklärung zur Bachelorarbeit}
\end{center}

\vspace{1cm}

\hspace*{-0.3cm}
\begin{tabular}{l l l}
    Name, Vorname Student{\ifnum\CAUTHORANR = 1 in\fi}: \CAUTHORNACH , \CAUTHORVOR&Matrikel-Nr.: \CMATRIKEL&Kurs: \CKURS\\
    \\
    Ausbildungsstätte: \CBETRIEB
\end{tabular}

\vspace{1.5cm}
Zur öffentlichen Einsichtnahme der Bachelorarbeit unsere{\ifnum\CAUTHORANR = 1 r\else s\fi} Student{\ifnum\CAUTHORANR = 1 in\else en\fi}, {\ifnum\CAUTHORANR = 1 Frau\else Herr\fi} \CAUTHOR , in der
Bibliothek der Dualen Hochschule Gera-Eisenach erklären wir uns: 
\vspace{1cm}

\hspace{2cm}$\square$ einverstanden

\hspace{2cm}$\square$ nicht einverstanden.

\vspace{1.5cm}
{\bf Thema der Bachelorarbeit:}

\vspace{1cm}
\CTITLE
\vspace{1cm}

\vspace*{\fill}
\begin{tabular} {lrl}
    \hspace{5.5cm} &  & \hspace{4cm} \\
    \hrulefill & & \hrulefill \\
    Datum, Stempel,  Unterschrift Firma/Einrichtung& & Datum, Unterschrift Student{\ifnum\CAUTHORANR = 1 in\fi}
\end{tabular}
\vspace*{\fill}

\fi

\end{document}
